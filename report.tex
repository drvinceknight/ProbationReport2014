\documentclass{article}

\title{Final Probation Report}
\author{Dr. Vincent Knight}

\usepackage{fullpage}
\usepackage[parfill]{parskip}
\usepackage{booktabs}
\usepackage{hyperref}

\begin{document}

\maketitle

\section{Introduction}\label{sec:introduction}

This document aims to evidence my accomplishments during the probationary period of my lectureship:

\begin{itemize}
    \item Section \ref{sec:teaching} will discuss my participation in PCUTL.
    Due to the volume my module 3 submission, I will keep this section brief.
    \item Section \ref{sec:research} will not only list my publications and grant income but also discuss my future plans for research.
    \item Section \ref{sec:innovation} will demonstrate my contribution to innovation and engagement as well as a detailing my administrative responsibilities.
    \item Section \ref{sec:additional} will demonstrate my competency with the Teaching and Research grade 7 profile.
\end{itemize}

\section{Teaching}\label{sec:teaching}

\subsection{PCUTL}

I started PCUTL (HMT007) in [FIND DATE HERE] and completed the 3rd module on [FIND DATE HERE]. I fully engaged with this process obtaining my certificate with distinction.

My module 3 portfolio is attached but due to the page count of that submission (FIND PAGE COUNT) is summarised below:

\begin{itemize}
    \item Rigorous statistical analysis of a student survey investigating student perceptions of formative assessment. A manuscript is in preparation to submit to a pedagogical journal.
    \item Detailed review and critique of active learning techniques including Inquiry Based Learning and Flipped Classrooms;
    \item Reflection within my subject area on the defining properties of a modern mathematician which should include not only entrepreneurial skills but also programming skills;
    \item Rigorous and justified implementation of a new module (MA1003) taught using an innovative and modern pedagogy (a flipped classroom delivered to a class of 160 students);
    \item Peer review by multiple Cardiff University colleagues, international peers and engagement with higher education academy.
\end{itemize}

The following summarises my present teaching philosophy which is a direct result of the PCUTL process:

\begin{quote}
   Aim to provide learning opportunities to students in a constructivist framework, using technology to enhance a scaffolded active student centred experience.
\end{quote}

It might be of interest to note that recent research (appearing after I completed PCUTL) has in fact shown the evidence for better student learning in active learning pedagogies as opposed to a classic lecture based approach \cite{freeman_active_2014}.

\subsection{Teaching}

As well as going through the PCUTL process I have been involved in various other teaching activities as summarised in Table \ref{tab:teaching}.

\begin{table}[!htbp]
\begin{center}
\begin{tabular}{p{5cm}llp{6cm}}
\toprule
Course Title & Credits & Level & Involvement\\
\midrule
Computing for Mathematics & 20 &  First Year BSc. & Designed, Lead and Delivered\\
OR 2 & 10 &  Final Year BSc.  & Designed and Delivered half of course\\
OR Methods & 12 & MSc. & Designed and Delivered 4/11 of course\\
Advanced Statistical Packages & 10 &  MSc. & Designed, Lead and Delivered\\
Introduction to Object Oriented Programming & NA &  MSc. & Designed, Lead and Delivered 2 day hackathon\\
Introduction to \LaTeX & NA & BSc. and MMath & Designed, Lead and Delivered half day course\\
\bottomrule
\end{tabular}
\caption{Summary of involvement in taught courses}\label{tab:teaching}
\end{center}
\end{table}

Most of the above courses are designed to be delivered in a student centred approach which is a direct implication of my growth as an educator through the PCUTL process.

This teaching has taken a large quantity of time in terms of preparation, the exact amount of time is difficult at this stage to approximate but the above corresponds to a mean of around 7 hours of contact time a week.

Future teaching plans involve the creation of an extra curricular 2 hour weekly session during which students will be able to further explore aspect of programming applied to mathematics: Code Club.

\subsection{Research Students}

Throughout my tenure as a lecturer I have been heavily involved in the supervision of research students as various levels:

\begin{itemize}
    \item BSc. Final Year Project ();
\end{itemize}


\section{Research and Scholarship}\label{sec:research}

\subsection{Publications}
\subsection{Grant Funding}
\subsection{Future Plans}

\section{Contribution to Innovation and Engagement}\label{sec:innovation}

\subsection{Innovation and Engagement}

\begin{itemize}
    \item School visits
    \item ORiS
    \item University visits
    \item Social media: (Twitter, G+, YouTube, Blog)
    \item Media
    \item HMC2 and Jenny.
    \item Open Source Software contributions.
\end{itemize}


\subsection{Administration}

\begin{itemize}
    \item Research committee;
    \item IT committee;
    \item General administration.
\end{itemize}

\section{Additional Requirements}\label{sec:additional}
\section{Conclusion}\label{sec:conclusion}

\newpage
\bibliographystyle{plain}
\bibliography{bibliography}

\end{document}
