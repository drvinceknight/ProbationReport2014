\documentclass{article}

\title{Final Probation Report}
\author{Dr. Vincent Knight}

\usepackage{fullpage}
\usepackage[parfill]{parskip}
\usepackage{booktabs}
\usepackage{hyperref}
\usepackage{etaremune}

\begin{document}

\maketitle

\section{Introduction}\label{sec:introduction}

This document aims to evidence my accomplishments during the probationary period of my lectureship:

\begin{itemize}
    \item Section \ref{sec:teaching} will discuss my participation in PCUTL.
    Due to the volume of my module 3 submission, I will keep this section brief.
    \item Section \ref{sec:research} will not only list my publications and grant income but also discuss my future plans for research.
    \item Section \ref{sec:innovation} will demonstrate my contribution to innovation and engagement as well as a detailing my administrative responsibilities.
    \item Section \ref{sec:additional} will demonstrate my competency with the Teaching and Research grade 7 profile.
\end{itemize}

\section{Teaching}\label{sec:teaching}

\subsection{PCUTL}

I started PCUTL (HMT007) in [FIND DATE HERE] and completed the 3rd module on [FIND DATE HERE]. I fully engaged with this process obtaining my certificate with distinction.

My module 3 portfolio is attached but due to the page count of that submission (FIND PAGE COUNT) is summarised below:

\begin{itemize}
    \item Rigorous statistical analysis of a student survey investigating student perceptions of formative assessment. A manuscript is in preparation to submit to a pedagogical journal.
    \item Detailed review and critique of active learning techniques including Inquiry Based Learning and Flipped Classrooms;
    \item Reflection within my subject area on the defining properties of a modern mathematician which should include not only entrepreneurial skills but also programming skills;
    \item Rigorous and justified implementation of a new module (MA1003) taught using an innovative and modern pedagogy (a flipped classroom delivered to a class of 160 students);
    \item Peer review by multiple Cardiff University colleagues, international peers and engagement with higher education academy.
\end{itemize}

The following summarises my present teaching philosophy which is a direct result of the PCUTL process:

\begin{quote}
   Aim to provide learning opportunities to students in a constructivist framework, using technology to enhance a scaffolded active student centred experience.
\end{quote}

It might be of interest to note that recent research (appearing after I completed PCUTL) has in fact shown the evidence for better student learning in active learning pedagogies as opposed to a classic lecture based approach \cite{freeman_active_2014}.

\subsection{Teaching}

As well as going through the PCUTL process I have been involved in various other teaching activities as summarised in Table \ref{tab:teaching}.

\begin{table}[!htbp]
\begin{center}
\begin{tabular}{p{5cm}llp{6cm}}
\toprule
Course Title & Credits & Level & Involvement\\
\midrule
Computing for Mathematics & 20 &  First Year BSc. & Designed, Lead and Delivered\\
OR 2 & 10 &  Final Year BSc.  & Designed and Delivered half of course\\
OR Methods & 12 & MSc. & Designed and Delivered 4/11 of course\\
Advanced Statistical Packages & 10 &  MSc. & Designed, Lead and Delivered\\
Introduction to Object Oriented Programming & NA &  MSc. & Designed, Lead and Delivered 2 day hackathon\\
Introduction to \LaTeX & NA & BSc. and MMath & Designed, Lead and Delivered half day course\\
\bottomrule
\end{tabular}
\caption{Summary of involvement in taught courses}\label{tab:teaching}
\end{center}
\end{table}

Most of the above courses are designed to be delivered in a student centred approach which is a direct implication of my growth as an educator through the PCUTL process.

This teaching has taken a large quantity of time in terms of preparation, the exact amount of time is difficult at this stage to approximate but the above corresponds to a mean of approximately 7 hours of contact time a week.

Future teaching plans involve the creation of an extra curricular 2 hour weekly session. During this students will be able to further explore aspect of programming applied to mathematics: Code Club.

\subsection{Research Students}

Throughout my tenure as a lecturer I have been heavily involved in the supervision of research students as various levels:

\begin{itemize}
    \item BSc. Final Year Students (10);
    \item MMath Final Year Student (1);
    \item Summer Research Students (7);
    \item PhD Students (5).
\end{itemize}

This aspect of teaching is something I am particular fond of and hope to continue.
In particular I hope to further enhance the involvement of undergraduates in research.

\subsection{Pedagogic Scholarship}

Further to the above I have fully engaged in pedagogic scholarship:

\begin{itemize}
    \item I have been the external examiner for the Statistics and Operational Research Program at the University of Greenwich since 2012. Due to my lack of experience this was done under the guidance of another external examiner.
    \item {I use my personal blog: \href{http://drvinceknight.blogspot.co.uk/}{`Un Peu de Math'} to write about research as well as mathematics education.
    The blog has more than 60,000 views and I estimate that 20,000 of those are on the subject of mathematics education.}
    \item I have spoken on the subject of mathematics education on multiple occasions including:
        \begin{itemize}
            \item PCUTL module 3 presentation (2013).
            \item Cardiff University special interest group on feedback (2013).
            \item Higher Education Academy STEM Education conference (2014).
        \end{itemize}
    \item I regularly run outreach activities in schools and am the chair of the OR Society task force: OR in Schools.
    \item In July 2014 Professor Harper and myself are hosting a workshop on innovative teaching approaches.
\end{itemize}

This engagement with the pedagogic community is a particular aspect of my job that I thoroughly enjoy and plan to continue to invest myself in.

\section{Research and Scholarship}\label{sec:research}

My research interests lie in the fields of Game Theory and Queueing Theory applied to Healthcare.
During my probationary period I have published 17 manuscripts in leading journals with one paper being returned to REF 2013.

\subsection{Publications}

A full list of my publications is given below. Of the 17 published papers 10 were published during my my probationary period.

/home/vince/CV/publications.tex

\subsection{Grant Funding}

Further to my publication portfolio I have had success in garnering funding as shown below:

\sl{Aneurin Bevan Health Board}\\
ABUHB Mathematical Modelling Research Unit\\
\pounds490,663 \hfill{2015-2017}


\sl{Cardiff University CUROP Award}\\
Building Game Theoretical Software in a Research Environment\\
\pounds1,800\hfill{2014}


\sl{Cardiff \& Vale University Health Board}\\
Operational Research Modelling to Support Cardiff and Vale UHB\\
\pounds 371,427 \hfill{2013-2018}


\sl{EPSRC}\\
Identifying and modelling victim, business regulatory and malware behaviours in a changing cyberthreat landscape\\
\pounds101,659\hfill{2013-2016}


\sl{Aneurin Bevan Health Board}\\
Creation of a Mathematical/OR Modelling Unit to Support the Aneurin Bevan Health Board\\
\pounds319,944 \hfill{2013-2015}


\sl{ESRC}\\
Hate speech? Understanding the modelling of social media identity formation and behaviour through the Cardiff Online Social Media Observatory (COSMOS)\\
\pounds7,015\hfill{2013-2016}


\sl{Health Foundation and Cardiff \& Vale University Health Board}\\
Estimating quality improvement and cost reduction for the patient and local health economy of transferring ENT/audiology services into a community setting\\
\pounds61,237\hfill{2013-2014}


\sl{LANCS (EPSRC)}\\
Post-Doctoral Training Scheme Grant: Investigating the Effects of Individual Behaviour on Hierarchical queueing Systems\\
\pounds5,000\hfill{2012}


\sl{Cardiff University CUROP Award}\\
Developing and Evaluating Mathematical Teaching Resources through Open Source Software\\
\pounds2,200\hfill{2012}

\sl{LANCS (EPSRC)}\\
Post-Doctoral Training Scheme Grant: Choice and Healthcare Investigation Project\\
\pounds2,500\hfill{2010-2011}

\sl{Cardiff University CUROP Award}\\
Patient Choice: A Discrete Event Simulation\\
\pounds2,500\hfill{2010}

\subsection{Future Plans}

I am proud of my research portfolio to date: gathering an international reputation in the fields of Game Theory, Queueing Theory and Healthcare modelling.
I have immediate plans for a number of papers that not only apply techniques to Healthcare modelling but also involve the theoretical understanding of behaviour in queues.

Some concrete examples include:

\begin{itemize}
    \item `On the solution of an equation that occurs in routing games';
    \item `Game Theoretic Analysis of CCU-EU interaction';
    \item `Equilibrium Behaviour in a series of two queues'.
\end{itemize}

Further to this, I am actively preparing my first grant proposal which will investigate the equilibrium behaviour and optimal control in queueing systems subject to time dependent variations.
This will not only involve a variety of mathematical techniques but also have a variety of applications in healthcare systems such as Emergency Units.

I also plan to investigate the potential for fellowships as well as continuing to grow an international network of collaborators (in 2013 I visited potential collaborators in Michigan).

\section{Contribution to Innovation and Engagement}\label{sec:innovation}

I am a very active contributor to the innovation and engagement agenda of the school of mathematics.

\subsection{Innovation and Engagement}

As discussed in Section \ref{sec:teaching} I regularly take part in outreach activities within local schools aiming to engage students and promote Cardiff University.
I have also regularly been an active participant in outreach activities taking place within the University.
This falls not only under my remit as a lecturer but also forms part of my responsibilities as Chair of the OR in Schools task force.

My outreach activities extend beyond the immediate physical proximity of the University as I have a very active web presence aiming to promote not only Operational Research and Mathematics but also Cardiff University:

\begin{itemize}
    \item Personal website \url{www.vincent-knight.com} with over 60 regular daily views.
    \item Google plus page with 26,000 followers and over 5,000,000 views of my content which include general discussions about Mathematics as well as the particular work done by the OR group within the School.
    \item YouTube channel (a combination of general outreach videos as well as teaching materials), over 114,000 views and 485 subscribers.
    \item A twitter account with 430 followers.
    \item My blog with over 60,000 pager views.
\end{itemize}

As well as Social media, I have also contributed to the University's exposure through traditional media outlets.
The ones that took place during my probationary period are listed below:

\begin{itemize}
    \item 2012-12-27 - Pythagoras' Trousers Christmas Show
    \item 2013-02-14 - Pythagoras' trousers appearance to talk about prime numbers
    \item 2014-03-19 - BBC Parliament - Voice of the Future 2014
    \item 2014-05-05 - 2014 Pythagoras Lecture: Mathematics and Healthcare Management
\end{itemize}

My outreach activities were recently recognised through the fact that I was a judge for the SET for Britain 2014 poster contribution which took place at Westminster.
Two aspects of my probationary period can be considered of particular value to the innovation agenda of the school:

\begin{itemize}
    \item I am involved with two healthcare modelling units that are innovative in nature by the fact that they embed post docs within locals health boards.
          Currently 5 post doctoral researchers are in post sharing their time between a Health Board and the School of Mathematics.
          This innovative setup ensures that novel mathematical research has an immediate impact on healthcare services.
          Funding for these centres was listed in Section \ref{sec:research}.
    \item I am an active contributor to open source projects and further more am the administrator for the Sage server that is a valuable teaching and learning tool for our students.
          Furthermore, this server has allowed for substantial financial savings with regards to licencing of mathematical software.
          I also contribute to the open source community through my involvement in the Django Cardiff user group.
          The most recent international conference organised by this group involved 3 student participants whose places were financed by the School of Mathematics.
\end{itemize}

\subsection{Administration}

As well as the various routine administrative duties (tutoring, meetings, time management etc...) I sit on the following administrative committees:

\begin{itemize}
    \item School Research committee;
    \item School IT committee;
    \item School Engagement committee;
    \item OR Society General council;
    \item OR Society Education and Research committee;
    \item OR Society Social Media committee.
\end{itemize}

\section{Additional Requirements}\label{sec:additional}

In this section I will describe how I already meet the various requirements of a Teaching and Research level 3 position.

\subsection{Communication}

I have experience of routinely communicating both complex and conceptual ideas to a variety of audiences:

\begin{itemize}
    \item Peers: I routinely present my work at international and prestigious conferences as well as publishing in highly ranked journals.
    \item Students: As stated in Section \ref{sec:teaching} I lecture on a variety of courses, obtaining high levels of feedback.
    \item General public: I have experience of communicating scientific ideas through the media as well as my YouTube channel and my blog.
\end{itemize}

\subsection{Team work and motivation}

I routinely am able to lead as well as follow within a team environment:

\begin{itemize}
    \item I have on multiple occasions advised and supported post doctoral researchers and PhD students with regards to personal development.
    A particular example of this is my position as line manager of the post doctoral researcher on the UHW grant (described in Section \ref{sec:research}).
    \item I have worked as part of a team developing productive work. One such example was the organisation of the ORAHS 2011 conference during which I had various responsibilities leading to an excellent conference.
    \item I have collaborated with colleagues to identify and respond to student needs in a multitude of ways.
    For example I have worked closely with the Cardiff University Enterprise team to develop and embed entrepreneurial skills within a mathematics module.
\end{itemize}

I have lead on various local research projects working with PhD students and post docs.
A particular example is the cross departmental group project for PCUTL.
For this project, I lead the quantitative analysis part of the project.
Finally, for a lecturer at my career stage I have a substantial experience of supervision of research teams.
These include:

\begin{itemize}
    \item Multiple BSc. projects;
    \item Specific research projects with post doctoral researchers;
    \item Multiple CUROP projects;
    \item Multiple PhD students.
\end{itemize}

\subsection{Liaison and networking}

I have already adopted various aspects of networking as a part of my responsibilities.
I regularly meet and seek funding with NHS partners (as evidenced by the grant income of Section \ref{sec:research}).
Furthermore I have started to gain an international network of potential collaborators including researchers at the University of Michigan.

With regards to pedagogic network, I have built a relationship with lecturer at the universities of X, Y, Z, who are experts on active learning methodologies.
These esteemed colleagues will be visiting the University for a workshop that myself and Professor Harper have organised.

I have been recently sought out by a student at the University of Clermont (France) to carry out a summer research project.

Finally, through my social media interactions and blog I regularly market Cardiff University as an exciting place to both study and do research.

\subsection{Service delivery}

I have some experience of handling service delivery requests as the relationship between the OR group and local health boards often entails one of requests for mathematical modelling.

I have on multiple occasions been able to deal with these requests and constantly portray a positive image of the institution by being responsive to requests.

\subsection{Decision making processes}

I match this criteria across multiple dimensions.
This is mainly due to the fact that I have designed 3 novel modules during my lectureship: being solely responsible for the design and delivery of these.
I have also on multiple occasions advised other member of staff with regards to potential ideas of generating income and promoting the subject: the modelling unit described in Section \ref{sec:research} has developed a teaching course for NHS members of staff.
Along with collaborators at the University of Exeter I had a role in the design and promotion of this course.

\subsection{Planning and organising resources}

Through my pedagogic approach based on an active student centred learning experience I have(as a module leader) had to develop a large quantity of learning resources that ensures that the expectations of students are met.

Furthermore, through the large workload I have carried throughout my lectureship I have demonstrated an excellent ability to manage projects relating my own area of work.

\subsection{Initiative and problem solving}

Through my role in leading research teams as well as developing novel modules I have shown an excellent sense of initiative and problem solving.
Specifically, I designed an innovative hackathon learning exercise following feedback from the MSc. advisory board.
This specific example not only shows my ability to identify the need for development of content of modules but also collaborate with colleagues on the implementation of assessment procedures.

Through my outreach and engagement activities I regularly disseminate and apply results of research and scholarship.
One example of this is my utilisation of Graph Theoretic methodologies to timetable an assessment activity for students.
The methodology used was disseminated to the relevant students as not only was the timetable of interest to them but the methodology used to obtain it was also of interest.

As a member of the engagement committee I regularly contribute to strategic issues regarding the marketing of the school of mathematics.

\subsection{Analysis and research}

My publication and grant funding record (see Section \ref{sec:research}) clearly demonstrate my ability to:

\begin{itemize}
    \item Develop research objectives, projects and proposals;
    \item Conduct individual and collaborative research projects;
    \item Identify sources of funding and contribute to the process of securing funds;
    \item Write publications and also disseminate research through my web presence;
\end{itemize}

I have presented at a variety of conferences and also delivered research driven outreach activities.

\subsection{Sensory and physical demands}

I have demonstrated through my extensive responsibilities an ability to handle the pressures of teaching, research and administrative demands under very tight deadlines.

\subsection{Work environment}

I am a trained first aider and also one of the fire wardens in the School.
This demonstrates my ability to take responsibility for the health and safety of others.

\subsection{Pastoral care and welfare}

Through my role as a module leader and a personal tutor I regularly demonstrate my ability to act responsibly for the pastoral care of my students.

\subsection{Team development}

I regularly supervise the work of others and ensure their continued personal development.
One example of this is my continued mentoring of an undergraduate research student who will be starting his PhD with my in October 2014.
Another example is in my role as deputy leader of the ABCi modelling group (described in Section \ref{sec:research}).

\subsection{Teaching and learning support}

As described in detail in Section \ref{sec:teaching} I have already taken on a large quantity of teaching responsibilities.

I have designed a range of teaching materials (see Table \ref{tab:teaching}) and delivered this using appropriate teaching, learning support and assessment methods.
I constantly reflect on my teaching activities and using various sources (for example student feedback) I have shown my ability to revise my teaching resources.
One example of this are the modifications made to my Computing for Mathematics course to better align with the expectations of students.

I have through my teaching responsibilities also demonstrated my ability to mark and asses work and examinations as well as provide feedback to students.

\subsection{Knowledge and experience}

\section{Conclusion}\label{sec:conclusion}

\newpage
\bibliographystyle{plain}
\bibliography{bibliography}

\end{document}
